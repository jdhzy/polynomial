\documentclass[11pt]{article}

\usepackage{fullpage}
\usepackage{enumitem}

%\usepackage{fancyhdr}
%\usepackage{times,graphicx,epstopdf,,amsfonts,amsthm,amsmath,xspace}
%\usepackage[left=.75in,top=.75in,right=.75in,bottom=.75in]{geometry}

\usepackage{amsmath,amsthm,amsfonts,amssymb,mathtools,latexsym}
\usepackage{xcolor,graphicx,hyperref,float}
\definecolor{oxblue}{RGB}{0, 33, 71}

\usepackage[ruled,vlined,linesnumbered]{algorithm2e}

% box around soln
\newcommand{\answerbox}[1]{\fbox{\textcolor{white}{\rule{#1}{24pt}}}}
\newcommand{\answerboxshort}[1]{\fbox{\textcolor{white}{\rule{#1}{14pt}}}}
\newcommand{\answerboxtall}[2]{\fbox{\textcolor{white}{\rule{#1}{#2}}}}


\newcommand\mycommfont[1]{\footnotesize\ttfamily\textcolor{blue}{#1}}
\SetCommentSty{mycommfont}


\pdfpagewidth 8.5in
\pdfpageheight 11in 

\oddsidemargin 0in
\evensidemargin 0in

\newtheorem{claim}{Claim}
\newtheorem{definition}{Definition}
\newtheorem{theorem}{Theorem}
\newtheorem{lemma}{Lemma}
\newtheorem{observation}{Observation}
\newtheorem{question}{Question}
\newtheorem{problem}{Problem}

%\newenvironment{proof}{\smallskip \noindent \emph{Proof:}} {\hfill $\Box$}

\newcommand{\re}{{\mathbb{R}}}
\newcommand{\floor}[1]{\lfloor {#1} \rfloor}
\newcommand{\ceil}[1]{\lceil {#1} \rceil}
\newcommand{\paren}[1]{\left( {#1} \right)}

\newenvironment{solution}
  {\medskip\noindent\color{oxblue}\textbf{Solution}. \small}
  {\medskip}

  \usepackage{listings}

  \definecolor{customgreen}{rgb}{0,0.6,0}
  \definecolor{customgray}{rgb}{0.5,0.5,0.5}
  \definecolor{custommauve}{rgb}{0.6,0,0.8}
  
  \lstset{ 
    basicstyle=\ttfamily\footnotesize,        % the size of the fonts that are used for the code
    breaklines=true,                 % sets automatic line breaking
    commentstyle=\color{customgreen},    % comment style
    firstnumber=1,                % start line enumeration with line 1000
    frame=single,	                   % adds a frame around the code
    keepspaces=true,                 % keeps spaces in text, useful for keeping indentation of code (possibly needs columns=flexible)
    keywordstyle=\color{blue},       % keyword style
    numbers=left,                    % where to put the line-numbers; possible values are (none, left, right)
    numbersep=10pt,                   % how far the line-numbers are from the code
    numberstyle=\tiny\color{customgray}, % the style that is used for the line-numbers
    rulecolor=\color{black},         % if not set, the frame-color may be changed on line-breaks within not-black text (e.g. comments (green here))
    showspaces=false,                % show spaces everywhere adding particular underscores; it overrides 'showstringspaces'
    showstringspaces=false,          % underline spaces within strings only
    showtabs=false,                  % show tabs within strings adding particular underscores
    stepnumber=1,                    % the step between two line-numbers. If it's 1, each line will be numbered
    stringstyle=\color{custommauve},     % string literal style
    tabsize=2,	                   % sets default tabsize to 2 spaces
    title=\lstname                   % show the filename of files included with \lstinputlisting; also try caption instead of title
  }
  

\begin{document}
	
\begin{center}
\Large{\textbf{CS 330, Fall 2024, Homework 2 Questions \\}}
\end{center}

\begin{center}
Due: Tuesday, 9/17, at 11:59 pm on Gradescope
\end{center}


\paragraph{Collaboration policy} Collaboration on homework problems is permitted,  you are allowed to discuss
each problem with at most 3 other students currently enrolled in the
class.
Before working with others on a problem, you should think about it
yourself for at least 45 minutes. Finding answers to problems on the
Web or from other outside sources (these include anyone not enrolled
in the class) is strictly forbidden.

{\em You must write up each problem solution by yourself without
	assistance, even if you collaborate with others to solve the
	problem.} You must also identify your collaborators. If you did not
work with anyone, you should write ``Collaborators: none.'' It is a
violation of this policy to submit a problem solution that you
cannot orally explain to an instructor or TA.

\paragraph{Typesetting} Solutions should be typed and submitted as a
PDF file on Gradescope. You may use any program you like to type your
solutions. \LaTeX, or ``Latex'', is commonly used for technical writing (\texttt{overleaf.com}
is a free web-based platform for writing in Latex) since it handles
math very well. Word, Google
Docs, Markdown or other software are also fine. 

\paragraph{Solution guidelines} For problems that require you to provide an algorithm, you must provide:
\begin{enumerate}
	\item pseudocode and, if helpful, a precise description of the algorithm in English. As always, pseudocode should include
  \begin{itemize}
		\item A clear description of the inputs and outputs
    \item Any assumptions you are making about the input (format, for example)
    \item Instructions that are clear enough that a classmate who
      hasn't thought about the problem yet would understand how to
      turn them into working code. Inputs and
      outputs of any subroutines should be clear, data structures
      should be explained, etc. 
  \end{itemize}
  \emph{If the algorithm is not clear enough for graders to understand easily, it may not be graded. }
	\item a proof of correctness,
		\item an analysis of running time and space.
\end{enumerate}
You may use algorithms from class as subroutines. You may also use facts that we proved in class.

You should be as clear and concise as possible in your write-up of
solutions. 

A simple, direct analysis is worth more points than a
convoluted one, both because it is simpler and less prone to error and
because it is easier to read and understand.

\newpage

\begin{enumerate}
    \item \textbf{Uniqueness of Stable Matching} (10 points).
   	Depending on hospitals' and residents' preferences, it could be that there is only one possible stable matching, or that there are many stable matchings. (As a warm-up exercise, find an example of preference lists with just one stable matching, and an example of preference lists with more than one stable matching.)
    
	Give an algorithm that takes preference lists for a set of $n$ hospitals and residents and decides if there is {\em exactly
          one} stable matching for this instance that is, the algorithm
        outputs either ``unique stable matching'', or ``more than one
        stable matching''.
        
     
        {\footnotesize \emph{Hints:} (\emph{a}) Read ``Extensions'' on page 9-12 of the book, Section 1.1. Use the fact that in the case that, the proposing side in the Gale-Shapley algorithms gets their best possible partner in any stable match, while the receiving side gets the worst possible partner in any stable match. 
        %
        (\emph{b}) When proving correctness, there are two things to prove. First, if the algorithm reports ``more than one
        stable matching'', then there really are multiple solutions; and second, if the algorithm reports ``unique stable matching'', there really is only one. }
        
    
    

    \item \textbf{Runtime Analysis} (10 points). BU's Computer Science department has decided to issue its own currency, called ``bucs'',  which comes in three denominations: 1, 2, and 3-buc coins. 
    
    Suppose you have three piles of coins: $a$ one-buc coins, $b$ two-buc coins, and $c$ three-buc coins. The question is this: if you want to give somebody $n$ bucs, how many different ways can you do it using the coins you have? All that matters is the number of coins of each type which you use, not the particular coins. 
    
    For example, if $a=3$, $b=2$, $c=2$  and $n=5$, there are 4 different ways you can choose coins: 

    {\footnotesize $ 3 \times \text{(one buc)} + 1 \times   \text{(two bucs)} $

    $ 2 \times \text{(one buc)} + 1 \times   \text{(three bucs)}$    

    $ 1 \times \text{(one buc)} + 2 \times   \text{(two bucs)}$    
    
        $ 1 \times \text{(two buc)} + 1 \times   \text{(three bucs)}$    }
        
        Your task is to write an algorithm \texttt{coinChoices} that takes positive integers  $a,b,c, n$ as input and outputs the number of ways to choose coins whose total value is $n$ (where $a,b,c$ are the limits on the number of coins of each type). So \texttt{coinChoices(3, 2, 2, 5)} should return 4. 


\begin{enumerate}
	\item Here is an algorithm, written as Python\footnote{You may use \textbf{any programming language} for this assignment, but you will have to implement the algorithm below in your chosen language.} code, that solves the problem!
\begin{lstlisting}[language=Python]
def coinChoices(a,b,c,n):
    assert type(a) == type(b) == type(c) == type(n) == int
    assert a> 0 and b>0 and c>0 and n>0
    max_i = min(a, n)
    max_j = min(b, n // 2)
    max_k = min(c, n // 3)
    count = 0
    for i in range(max_i+1): 
        for j in range(max_j+1):
            for k in range(max_k+1): 
                if i + 2 * j + 3 * k == n:
                    count += 1
                    # print ( (i, j, k), "is a way to get value", n)
    return count
\end{lstlisting}

	Run this Python function on inputs of the form $(n, n, n, n)$ for values of $n$ that are powers of 2 ranging from 16 to 2048. Measure how much time each execution takes. For example, you could use code like this:
	
\begin{lstlisting}[language=Python]
import time
powersoftwo =  [2 ** x for x in range(4,12)]
print(powersoftwo)
run_times = []
for n in powersoftwo:
    tick = time.perf_counter()
    print("n =", n)
    coinChoices(n, n, n, n)
    tock = time.perf_counter()
    print(tock-tick, "seconds")
    run_times.append(tock-tick) 
 \end{lstlisting}

Plot the running times you get (using Python, Excel, or any other software), using a \textbf{logarithmic scale for the axes} (a ``log-log'' plot\footnote{If you are not familiar with log plots, skim the Wikipedia article on ``Logarithmic Scale''.}). You should see a roughly straight  line! Now trying comparing the values you got to a function of the form $y = w n^3$—that is, try to find a value $w$ such that the run times you observed are closely matched by  $y = w n^3$.  (On one laptop, $w = 5 \times 10^{-8}$ worked well, but your hardware will vary.)

For example, you could use Python code like this: 
\begin{lstlisting}[language=Python]
from matplotlib import pyplot as plt
w = 4 * 10 ** -8
cubes = [w * n ** 3 for n in powersoftwo]
plt.plot(powersoftwo, run_times, "r.-")
plt.plot(powersoftwo, cubes, "b--")
plt.yscale('log')
plt.xscale('log')
plt.xlabel("n")
plt.ylabel("time (s)")
plt.show()
 \end{lstlisting}

	For this part, just submit the value $w$ that you got and the resulting plot. If you were not able to find such a $w$, explain what you think might have gone wrong.  
	
    \item Give an asymptotically faster algorithm for this problem. (See guidelines: include pseudocode, a proof, and a run time analysis.) 
    For the run time analysis, give a bound as a function of $n$, assuming $a,b,c$ are at most $n$ (your code should be correct for all such choices of $a,b,c$, as should your running time analysis). 
    Any algorithm with running time $O(n^2)$ will get full credit (better is great). How fast an algorithm can you find?
    
    \item Test your runtime analysis experimentally by coding up your algorithm and plotting its running time (as you did above), and seeing if you can approximately match the times with a curve of the form $y = w F(n)$ where $F(n)$ is the asymptotic running time you got in the previous part. 
    
    Include your plot with two curves: the running time of your algorithm, and that of the best function you could find of the form $y = w F(n)$. Include a brief discussion—a few sentences—of how you chose $t$ and any problems you encountered.
        \end{enumerate}



\end{enumerate}
	
\end{document}


